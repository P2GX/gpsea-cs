\begin{table}
\centering
\renewcommand{\arraystretch}{1.2} % Optional: Increase row spacing
\begin{tabular}{l>{\raggedright\arraybackslash}p{6cm}>{\raggedright\arraybackslash}p{6cm}}
\toprule
\textbf{Name} & \textbf{Match criterion} & \textbf{Example}\\
\midrule
variant key & Specific variant & 1\_8364773\_8364773\_A\_AC (NM\_001042681.2:c.1512dup) \\
variant effect & Leads to the target effect & START\_LOST \\
variant class & Variant has the target type & SNV, DEL, DUP, INS, INV \\
gene & Variant affects gene & Affects \textit{RERE} \\
transcript & Variant affects transcript & Overlaps NM\_12345.01 \\
exon & Variant overlaps exon & Located in exon 3 \\
region & Variant overlaps region on protein & Located between amino acid residues 1 to 77 \\
is large imprecise sv & Variant is structural variant without exact breakpoint coordinates & -- \\
is structural variant & The variant affects at least n base pairs (n=50 by default) or is a large imprecise SV or a translocation & -- \\
structural type & The target ontology class for a variant with imprecise/unknown breakpoints & Chromosomal deletion (SO:1000029) \\
is structural deletion & Variant is a Chromosomal deletion (SO:1000029) or an SV with known breakpoints that deletes at least a $n$ base pairs (\texttt{n=50bp} by default) & -- \\
ref length & Length of the reference sequence is above, below, or (not) equal to $n$ bases & $n>50$ for variants affecting more than 50 bp \\
change length & Change of length between the REF/ALT is above, below, or (not) equal to \texttt{n} bases & $n\leq-50$ for testing if a variant removes at least 50 base pairs  \\
protein feature type & Variant affects protein feature type & DOMAIN \\
protein feature & Variant region overlaps a protein feature & Overlaps with "Bipartite nuclear localization signal" \\
allof & Logical AND for two or more predicates to test if \textit{all} predicates match & missense AND exon 7 \\
anyof & Logical OR for two or more predicates to test if \textit{any} predicate matches & START\_LOST OR STOP\_GAINED \\
not & Logical NOT & NOT(missense) \\
\bottomrule
\end{tabular}
\caption{\textbf{Variant predicates.} Version 0.9.4 of GPSEA offers the variant predicates shown in this table. Each predicate evaluates a variant and returns \texttt{True} or \texttt{False}.}
\label{tab:varpredicates}
\end{table}

%%%%%%%%%%%%%%%%%%%%%%%%%%%%%%%%%%%%%%
\clearpage
\newpage

%%%%%%%%%%%%%%%%%%%%%%%%%%%%%%%%%%%%%%
\begin{center}
\begin{scriptsize}
\begin{longtable}
{l>{\raggedright}p{2.5cm}>{\raggedright}p{1.5cm}l>{\raggedright}p{1.5cm}lll}
\caption{Fischer exact test for association between genotypes and phenotypic features. $\complement$: set complement of a variant predicate. E.g. $\complement$ for a “missense” predicate includes any mutation that is \textit{not} missense.
}\label{tab:hpo_fet} \\
\hline
\multicolumn{1}{c}{\textbf{Cohort}} & \multicolumn{1}{c}{\textbf{HPO}} & \multicolumn{1}{c}{\textbf{Genotype A}} & \multicolumn{1}{c}{\textbf{}} & \multicolumn{1}{c}{\textbf{Genotype B}} & \multicolumn{1}{c}{\textbf{}} & \multicolumn{1}{c}{\textbf{p-val}} & \multicolumn{1}{c}{\textbf{adj. p}}\\ \hline
\endfirsthead
\multicolumn{8}{c}%

{{\bfseries \tablename\ \thetable{} -- continued from previous page}} \\ 

\hline
\multicolumn{1}{c}{\textbf{Cohort}} & \multicolumn{1}{c}{\textbf{HPO}} & \multicolumn{1}{c}{\textbf{Genotype A}} & \multicolumn{1}{c}{\textbf{}} & \multicolumn{1}{c}{\textbf{Genotype B}} & \multicolumn{1}{c}{\textbf{}} & \multicolumn{1}{c}{\textbf{p-val}} & \multicolumn{1}{c}{\textbf{adj. p}}\\ \hline
\endhead 
\hline \multicolumn{8}{|r|}{{Continued on next page}} \\ \hline 

\endfoot 
\hline \hline 
\endlastfoot 
BRD4 & Intrauterine growth retardation [HP:0001511] & NIPBL & 37/45 (82\%) & BRD4 & 2/9 (22\%) & $9.4 \times 10^{-04}$ & 0.032\\
EHMT1 & Attention deficit hyperactivity disorder [HP:0007018] & N Term Frameshift & 4/8 (50\%) & $^{\complement}$ & 3/96 (3\%) & $4.8 \times 10^{-04}$ & 0.029\\
FBN1 & Disproportionate tall stature [HP:0001519] & TB domain & 2/40 (5\%) & cbEGF & 12/43 (28\%) & 0.007 & 0.029\\
FBN1 & Ectopia lentis [HP:0001083] & TB domain & 9/18 (50\%) & cbEGF & 48/59 (81\%) & 0.013 & 0.038\\
FBN1 & Mitral valve prolapse [HP:0001634] & TB domain & 1/28 (4\%) & cbEGF & 13/47 (28\%) & 0.012 & 0.038\\
FBN1 & Proportionate short stature [HP:0003508] & TB domain & 20/36 (56\%) & cbEGF & 0/24 (0\%) & $2.3 \times 10^{-06}$ & $2.3 \times 10^{-05}$\\
FBN1 & Severe short stature [HP:0003510] & TB domain & 15/36 (42\%) & cbEGF & 0/24 (0\%) & $1.4 \times 10^{-04}$ & $9.0 \times 10^{-04}$\\
FBN1 & Short stature [HP:0004322] & TB domain & 23/39 (59\%) & cbEGF & 0/24 (0\%) & $4.4 \times 10^{-07}$ & $8.8 \times 10^{-06}$\\
FBN1 & Tall stature [HP:0000098] & TB domain & 7/38 (18\%) & cbEGF & 21/40 (52\%) & 0.002 & 0.011\\
FBN1 & Arachnodactyly [HP:0001166] & exon 37 & 0/8 (0\%) & $^{\complement}$ & 52/95 (55\%) & 0.003 & 0.019\\
FBN1 & Ectopia lentis [HP:0001083] & exon 37 & 1/9 (11\%) & $^{\complement}$ & 78/100 (78\%) & $1.1 \times 10^{-04}$ & $1.0 \times 10^{-03}$\\
FBN1 & Stiff skin [HP:0030053] & exon 37 & 8/9 (89\%) & $^{\complement}$ & 0/50 (0\%) & $4.1 \times 10^{-09}$ & $8.5 \times 10^{-08}$\\
FBN1 & Hyperextensibility of the finger joints [HP:0001187] & fs last two & 2/2 (100\%) & $^{\complement}$ & 1/90 (1\%) & $7.2 \times 10^{-04}$ & 0.012\\
FBN1 & Arachnodactyly [HP:0001166] & missense & 34/81 (42\%) & $^{\complement}$ & 18/22 (82\%) & $1.0 \times 10^{-03}$ & 0.026\\
FBN1 & Hyperextensibility of the finger joints [HP:0001187] & missense & 0/78 (0\%) & $^{\complement}$ & 3/14 (21\%) & 0.003 & 0.026\\
FBN1 & Thoracic aortic aneurysm [HP:0012727] & missense & 25/64 (39\%) & $^{\complement}$ & 12/14 (86\%) & 0.002 & 0.026\\
FBXL4 & Feeding difficulties [HP:0011968] & missense/ $^{\complement}$ OR $^{\complement}$/ $^{\complement}$ & 23/24 (96\%) & missense/ missense & 13/27 (48\%) & $1.7 \times 10^{-04}$ & 0.010\\
FGD1 & Broad foot [HP:0001769] & missense & 1/7 (14\%) & $^{\complement}$ & 14/15 (93\%) & $6.2 \times 10^{-04}$ & 0.032\\
GLI3 & Anal atresia [HP:0002023] & Fs in mid region or splice & 7/25 (28\%) & $^{\complement}$ & 3/57 (5\%) & 0.007 & 0.036\\
GLI3 & Nail dysplasia [HP:0002164] & Fs in mid region or splice & 7/20 (35\%) & $^{\complement}$ & 2/54 (4\%) & $1.0 \times 10^{-03}$ & 0.009\\
GLI3 & Preaxial foot polydactyly [HP:0001841] & Fs in mid region or splice & 5/25 (20\%) & $^{\complement}$ & 32/57 (56\%) & 0.003 & 0.021\\
GLI3 & Y-shaped metacarpals [HP:0006042] & Fs in mid region or splice & 11/23 (48\%) & $^{\complement}$ & 4/56 (7\%) & $9.9 \times 10^{-05}$ & $1.0 \times 10^{-03}$\\
GLI3 & Y-shaped metatarsals [HP:0010567] & Fs in mid region or splice & 11/19 (58\%) & $^{\complement}$ & 4/50 (8\%) & $3.2 \times 10^{-05}$ & $7.8 \times 10^{-04}$\\
GLI3 & Anal atresia [HP:0002023] & Truncating variants in Exon 15 & 8/28 (29\%) & $^{\complement}$ & 2/54 (4\%) & 0.002 & 0.017\\
GLI3 & Macrocephaly [HP:0000256] & Truncating variants in Exon 15 & 13/16 (81\%) & $^{\complement}$ & 15/42 (36\%) & 0.003 & 0.017\\
GLI3 & Postaxial foot polydactyly [HP:0001830] & Truncating variants in Exon 15 & 9/20 (45\%) & $^{\complement}$ & 2/35 (6\%) & $8.9 \times 10^{-04}$ & 0.011\\
GLI3 & Preaxial foot polydactyly [HP:0001841] & Truncating variants in Exon 15 & 7/28 (25\%) & $^{\complement}$ & 30/54 (56\%) & 0.010 & 0.050\\
GLI3 & Syndactyly [HP:0001159] & Truncating variants in Exon 15 & 5/17 (29\%) & $^{\complement}$ & 33/38 (87\%) & $4.8 \times 10^{-05}$ & $1.0 \times 10^{-03}$\\
GLI3 & Macrocephaly [HP:0000256] & Variants in C-terminal third & 13/15 (87\%) & $^{\complement}$ & 15/43 (35\%) & $7.3 \times 10^{-04}$ & 0.004\\
GLI3 & Postaxial foot polydactyly [HP:0001830] & Variants in C-terminal third & 9/16 (56\%) & $^{\complement}$ & 2/39 (5\%) & $7.3 \times 10^{-05}$ & 0.002\\
GLI3 & Postaxial hand polydactyly [HP:0001162] & Variants in C-terminal third & 15/16 (94\%) & $^{\complement}$ & 21/49 (43\%) & $3.4 \times 10^{-04}$ & 0.003\\
GLI3 & Syndactyly [HP:0001159] & Variants in C-terminal third & 5/16 (31\%) & $^{\complement}$ & 33/39 (85\%) & $2.2 \times 10^{-04}$ & 0.003\\
IKZF1 & Recurrent pneumonia [HP:0006532] & p.Asn159Ser & 7/9 (78\%) & $^{\complement}$ & 7/39 (18\%) & $1.0 \times 10^{-03}$ & 0.035\\
IKZF1 & T-cell acute lymphoblastic leukemias [HP:0006727] & p.Asn159Ser & 3/13 (23\%) & $^{\complement}$ & 0/69 (0\%) & 0.003 & 0.047\\
ITPR1 & Aniridia [HP:0000526] & GS Hotspot & 14/20 (70\%) & $^{\complement}$ & 6/48 (12\%) & $6.1 \times 10^{-06}$ & $1.6 \times 10^{-04}$\\
ITPR1 & Delayed ability to sit [HP:0025336] & GS Hotspot & 13/13 (100\%) & $^{\complement}$ & 40/68 (59\%) & 0.003 & 0.030\\
ITPR1 & Delayed ability to walk [HP:0031936] & GS Hotspot & 14/14 (100\%) & $^{\complement}$ & 40/67 (60\%) & 0.003 & 0.030\\
ITPR1 & Delayed gross motor development [HP:0002194] & GS Hotspot & 18/18 (100\%) & $^{\complement}$ & 57/83 (69\%) & 0.005 & 0.032\\
ITPR1 & Motor delay [HP:0001270] & GS Hotspot & 19/19 (100\%) & $^{\complement}$ & 68/94 (72\%) & 0.006 & 0.032\\
ITPR1 & Aniridia [HP:0000526] & IP3 binding & 0/14 (0\%) & $^{\complement}$ & 20/54 (37\%) & 0.007 & 0.024\\
ITPR1 & Delayed ability to sit [HP:0025336] & IP3 binding & 11/11 (100\%) & $^{\complement}$ & 42/70 (60\%) & 0.013 & 0.043\\
ITPR1 & Delayed ability to walk [HP:0031936] & IP3 binding & 13/13 (100\%) & $^{\complement}$ & 41/68 (60\%) & 0.004 & 0.019\\
ITPR1 & Delayed gross motor development [HP:0002194] & IP3 binding & 18/18 (100\%) & $^{\complement}$ & 57/83 (69\%) & 0.005 & 0.024\\
ITPR1 & Delayed speech and language development [HP:0000750] & IP3 binding & 13/13 (100\%) & $^{\complement}$ & 37/63 (59\%) & 0.003 & 0.019\\
ITPR1 & Motor delay [HP:0001270] & IP3 binding & 24/24 (100\%) & $^{\complement}$ & 63/89 (71\%) & 0.002 & 0.019\\
ITPR1 & Neurodevelopmental delay [HP:0012758] & IP3 binding & 29/29 (100\%) & $^{\complement}$ & 70/90 (78\%) & 0.003 & 0.019\\
ITPR1 & Nystagmus [HP:0000639] & IP3 binding & 25/26 (96\%) & $^{\complement}$ & 53/78 (68\%) & 0.003 & 0.019\\
ITPR1 & Delayed ability to sit [HP:0025336] & SV Deletion & 0/19 (0\%) & $^{\complement}$ & 53/62 (85\%) & $4.6 \times 10^{-12}$ & $1.7 \times 10^{-11}$\\
ITPR1 & Delayed ability to walk [HP:0031936] & SV Deletion & 0/19 (0\%) & $^{\complement}$ & 54/62 (87\%) & $1.5 \times 10^{-12}$ & $7.6 \times 10^{-12}$\\
ITPR1 & Delayed gross motor development [HP:0002194] & SV Deletion & 0/19 (0\%) & $^{\complement}$ & 75/82 (91\%) & $4.0 \times 10^{-15}$ & $3.0 \times 10^{-14}$\\
ITPR1 & Delayed speech and language development [HP:0000750] & SV Deletion & 0/19 (0\%) & $^{\complement}$ & 50/57 (88\%) & $1.7 \times 10^{-12}$ & $7.6 \times 10^{-12}$\\
ITPR1 & Global developmental delay [HP:0001263] & SV Deletion & 0/19 (0\%) & $^{\complement}$ & 48/57 (84\%) & $1.8 \times 10^{-11}$ & $5.7 \times 10^{-11}$\\
ITPR1 & Motor delay [HP:0001270] & SV Deletion & 0/19 (0\%) & $^{\complement}$ & 87/94 (93\%) & $3.9 \times 10^{-16}$ & $4.3 \times 10^{-15}$\\
ITPR1 & Neurodevelopmental delay [HP:0012758] & SV Deletion & 0/19 (0\%) & $^{\complement}$ & 99/100 (99\%) & $4.1 \times 10^{-21}$ & $9.0 \times 10^{-20}$\\
ITPR1 & Nystagmus [HP:0000639] & SV Deletion & 17/17 (100\%) & $^{\complement}$ & 61/87 (70\%) & 0.006 & 0.016\\
KCNH5 & Epileptic encephalopathy [HP:0200134] & Arg327His & 15/15 (100\%) & Arg333His & 0/3 (0\%) & $1.0 \times 10^{-03}$ & 0.032\\
KDM6A & Pulmonic stenosis [HP:0001642] & p.Asn891ValfsTer27 & 2/2 (100\%) & $^{\complement}$ & 0/59 (0\%) & $5.5 \times 10^{-04}$ & 0.016\\
LMNA & Distal muscle weakness [HP:0002460] & Gly608= & 0/15 (0\%) & $^{\complement}$ & 36/97 (37\%) & 0.002 & 0.004\\
LMNA & Elevated hemoglobin A1c [HP:0040217] & Gly608= & 0/15 (0\%) & $^{\complement}$ & 79/110 (72\%) & $5.7 \times 10^{-08}$ & $6.8 \times 10^{-07}$\\
LMNA & Limb muscle weakness [HP:0003690] & Gly608= & 0/15 (0\%) & $^{\complement}$ & 38/99 (38\%) & 0.002 & 0.004\\
LMNA & Lipodystrophy [HP:0009125] & Gly608= & 15/15 (100\%) & $^{\complement}$ & 127/236 (54\%) & $1.9 \times 10^{-04}$ & $7.5 \times 10^{-04}$\\
LMNA & Muscle weakness [HP:0001324] & Gly608= & 0/15 (0\%) & $^{\complement}$ & 63/124 (51\%) & $6.1 \times 10^{-05}$ & $3.6 \times 10^{-04}$\\
LMNA & Proximal muscle weakness [HP:0003701] & Gly608= & 0/15 (0\%) & $^{\complement}$ & 53/114 (46\%) & $3.6 \times 10^{-04}$ & $1.0 \times 10^{-03}$\\
LMNA & Proximal muscle weakness in upper limbs [HP:0008997] & Gly608= & 0/15 (0\%) & $^{\complement}$ & 35/102 (34\%) & 0.005 & 0.007\\
LMNA & Upper limb muscle weakness [HP:0003484] & Gly608= & 0/15 (0\%) & $^{\complement}$ & 35/96 (36\%) & 0.003 & 0.004\\
LMNA & Achilles tendon contracture [HP:0001771] & Upstream Tail (1-383) & 17/81 (21\%) & $^{\complement}$ & 23/34 (68\%) & $3.4 \times 10^{-06}$ & $1.4 \times 10^{-05}$\\
LMNA & Atrioventricular block [HP:0001678] & Upstream Tail (1-383) & 17/35 (49\%) & $^{\complement}$ & 8/117 (7\%) & $1.2 \times 10^{-07}$ & $1.5 \times 10^{-06}$\\
LMNA & Dilated cardiomyopathy [HP:0001644] & Upstream Tail (1-383) & 35/69 (51\%) & $^{\complement}$ & 5/103 (5\%) & $2.2 \times 10^{-12}$ & $4.2 \times 10^{-11}$\\
LMNA & Elbow contracture [HP:0034391] & Upstream Tail (1-383) & 17/79 (22\%) & $^{\complement}$ & 22/30 (73\%) & $9.6 \times 10^{-07}$ & $7.3 \times 10^{-06}$\\
LMNA & First degree atrioventricular block [HP:0011705] & Upstream Tail (1-383) & 7/62 (11\%) & $^{\complement}$ & 2/116 (2\%) & 0.009 & 0.032\\
LMNA & Foot joint contracture [HP:0008366] & Upstream Tail (1-383) & 17/77 (22\%) & $^{\complement}$ & 23/31 (74\%) & $6.7 \times 10^{-07}$ & $6.4 \times 10^{-06}$\\
LMNA & Hip contracture [HP:0003273] & Upstream Tail (1-383) & 4/79 (5\%) & $^{\complement}$ & 9/30 (30\%) & $1.0 \times 10^{-03}$ & 0.004\\
LMNA & Limb joint contracture [HP:0003121] & Upstream Tail (1-383) & 19/79 (24\%) & $^{\complement}$ & 23/31 (74\%) & $2.1 \times 10^{-06}$ & $9.8 \times 10^{-06}$\\
LMNA & Lipodystrophy [HP:0009125] & Upstream Tail (1-383) & 9/88 (10\%) & $^{\complement}$ & 133/163 (82\%) & $1.7 \times 10^{-29}$ & $6.5 \times 10^{-28}$\\
LMNA & Lower-limb joint contracture [HP:0005750] & Upstream Tail (1-383) & 18/78 (23\%) & $^{\complement}$ & 23/31 (74\%) & $1.5 \times 10^{-06}$ & $8.4 \times 10^{-06}$\\
LMNA & Second degree atrioventricular block [HP:0011706] & Upstream Tail (1-383) & 6/67 (9\%) & $^{\complement}$ & 1/116 (1\%) & 0.010 & 0.032\\
LMNA & Upper-limb joint contracture [HP:0100360] & Upstream Tail (1-383) & 17/77 (22\%) & $^{\complement}$ & 22/30 (73\%) & $1.4 \times 10^{-06}$ & $8.4 \times 10^{-06}$\\
LMNA & Achilles tendon contracture [HP:0001771] & Upstream of NLS & 21/85 (25\%) & $^{\complement}$ & 19/30 (63\%) & $2.7 \times 10^{-04}$ & $1.0 \times 10^{-03}$\\
LMNA & Atrioventricular block [HP:0001678] & Upstream of NLS & 17/36 (47\%) & $^{\complement}$ & 8/116 (7\%) & $2.1 \times 10^{-07}$ & $2.7 \times 10^{-06}$\\
LMNA & Dilated cardiomyopathy [HP:0001644] & Upstream of NLS & 35/70 (50\%) & $^{\complement}$ & 5/102 (5\%) & $4.2 \times 10^{-12}$ & $8.0 \times 10^{-11}$\\
LMNA & Elbow contracture [HP:0034391] & Upstream of NLS & 21/83 (25\%) & $^{\complement}$ & 18/26 (69\%) & $1.0 \times 10^{-04}$ & $6.6 \times 10^{-04}$\\
LMNA & First degree atrioventricular block [HP:0011705] & Upstream of NLS & 7/63 (11\%) & $^{\complement}$ & 2/115 (2\%) & 0.010 & 0.033\\
LMNA & Foot joint contracture [HP:0008366] & Upstream of NLS & 21/81 (26\%) & $^{\complement}$ & 19/27 (70\%) & $6.2 \times 10^{-05}$ & $5.9 \times 10^{-04}$\\
LMNA & Hip contracture [HP:0003273] & Upstream of NLS & 4/83 (5\%) & $^{\complement}$ & 9/26 (35\%) & $2.7 \times 10^{-04}$ & $1.0 \times 10^{-03}$\\
LMNA & Limb joint contracture [HP:0003121] & Upstream of NLS & 23/83 (28\%) & $^{\complement}$ & 19/27 (70\%) & $1.7 \times 10^{-04}$ & $8.1 \times 10^{-04}$\\
LMNA & Lipodystrophy [HP:0009125] & Upstream of NLS & 10/93 (11\%) & $^{\complement}$ & 132/158 (84\%) & $1.7 \times 10^{-31}$ & $6.6 \times 10^{-30}$\\
LMNA & Lower-limb joint contracture [HP:0005750] & Upstream of NLS & 22/82 (27\%) & $^{\complement}$ & 19/27 (70\%) & $8.2 \times 10^{-05}$ & $6.2 \times 10^{-04}$\\
LMNA & Pancreatitis [HP:0001733] & Upstream of NLS & 4/7 (57\%) & $^{\complement}$ & 14/110 (13\%) & 0.011 & 0.033\\
LMNA & Second degree atrioventricular block [HP:0011706] & Upstream of NLS & 6/68 (9\%) & $^{\complement}$ & 1/115 (1\%) & 0.011 & 0.033\\
LMNA & Upper-limb joint contracture [HP:0100360] & Upstream of NLS & 21/81 (26\%) & $^{\complement}$ & 18/26 (69\%) & $1.2 \times 10^{-04}$ & $6.7 \times 10^{-04}$\\
LMNA & Elevated hemoglobin A1c [HP:0040217] & missense & 73/101 (72\%) & $^{\complement}$ & 6/24 (25\%) & $3.1 \times 10^{-05}$ & $5.9 \times 10^{-04}$\\
LMNA & Loss of truncal subcutaneous adipose tissue [HP:0009002] & missense & 104/104 (100\%) & $^{\complement}$ & 4/11 (36\%) & $7.5 \times 10^{-09}$ & $2.9 \times 10^{-07}$\\
MPV17 & Peripheral axonal neuropathy [HP:0003477] & Pro98Leu/ Pro98Leu & 3/3 (100\%) & $^{\complement}$/$^{\complement}$ OR Pro98Leu/$^{\complement}$ & 1/22 (5\%) & 0.002 & 0.037\\
NBAS & Decreased circulating IgG concentration [HP:0004315] & missense/ missense & 1/12 (8\%) & $^{\complement}$/ $^{\complement}$ OR missense/ $^{\complement}$ & 15/23 (65\%) & 0.002 & 0.035\\
NF1 & Axillary freckling [HP:0000997] & Leu847Pro & 50/61 (82\%) & $^{\complement}$ & 75/193 (39\%) & $3.5 \times 10^{-09}$ & $7.8 \times 10^{-08}$\\
NF1 & Freckling [HP:0001480] & Leu847Pro & 54/54 (100\%) & $^{\complement}$ & 223/301 (74\%) & $5.3 \times 10^{-07}$ & $5.8 \times 10^{-06}$\\
NF1 & Inguinal freckling [HP:0030052] & Leu847Pro & 34/60 (57\%) & $^{\complement}$ & 58/189 (31\%) & $3.9 \times 10^{-04}$ & 0.002\\
NF1 & Lisch nodules [HP:0009737] & Leu847Pro & 22/42 (52\%) & $^{\complement}$ & 61/242 (25\%) & $7.5 \times 10^{-04}$ & 0.003\\
NF1 & Optic nerve glioma [HP:0009734] & Leu847Pro & 15/33 (45\%) & $^{\complement}$ & 24/143 (17\%) & $8.6 \times 10^{-04}$ & 0.003\\
NF1 & Plexiform neurofibroma [HP:0009732] & Leu847Pro & 24/66 (36\%) & $^{\complement}$ & 44/329 (13\%) & $4.6 \times 10^{-05}$ & $3.4 \times 10^{-04}$\\
NF1 & Axillary freckling [HP:0000997] & Met992del & 0/14 (0\%) & $^{\complement}$ & 125/240 (52\%) & $8.6 \times 10^{-05}$ & $1.0 \times 10^{-03}$\\
NF1 & Inguinal freckling [HP:0030052] & Met992del & 0/14 (0\%) & $^{\complement}$ & 92/235 (39\%) & 0.003 & 0.008\\
NF1 & Lipoma [HP:0012032] & Met992del & 5/44 (11\%) & $^{\complement}$ & 4/284 (1\%) & 0.003 & 0.008\\
NF1 & Lisch nodules [HP:0009737] & Met992del & 3/36 (8\%) & $^{\complement}$ & 80/248 (32\%) & 0.003 & 0.008\\
NF1 & Plexiform neurofibroma [HP:0009732] & Met992del & 0/44 (0\%) & $^{\complement}$ & 68/351 (19\%) & $2.0 \times 10^{-04}$ & $1.0 \times 10^{-03}$\\
NF1 & Scoliosis [HP:0002650] & SV & 18/57 (32\%) & $^{\complement}$ & 38/308 (12\%) & $9.7 \times 10^{-04}$ & 0.017\\
NF1 & Axillary freckling [HP:0000997] & p.Arg1830 & 20/79 (25\%) & $^{\complement}$ & 105/175 (60\%) & $3.7 \times 10^{-07}$ & $2.0 \times 10^{-06}$\\
NF1 & Freckling [HP:0001480] & p.Arg1830 & 76/133 (57\%) & $^{\complement}$ & 201/222 (91\%) & $6.3 \times 10^{-13}$ & $6.9 \times 10^{-12}$\\
NF1 & Inguinal freckling [HP:0030052] & p.Arg1830 & 8/79 (10\%) & $^{\complement}$ & 84/170 (49\%) & $4.9 \times 10^{-10}$ & $3.6 \times 10^{-09}$\\
NF1 & Lisch nodules [HP:0009737] & p.Arg1830 & 11/91 (12\%) & $^{\complement}$ & 72/193 (37\%) & $6.6 \times 10^{-06}$ & $2.9 \times 10^{-05}$\\
NF1 & Neurofibroma [HP:0001067] & p.Arg1830 & 5/8 (62\%) & $^{\complement}$ & 153/162 (94\%) & 0.012 & 0.030\\
NF1 & Optic nerve glioma [HP:0009734] & p.Arg1830 & 0/39 (0\%) & $^{\complement}$ & 39/137 (28\%) & $1.7 \times 10^{-05}$ & $6.3 \times 10^{-05}$\\
NF1 & Plexiform neurofibroma [HP:0009732] & p.Arg1830 & 0/135 (0\%) & $^{\complement}$ & 68/260 (26\%) & $1.7 \times 10^{-14}$ & $3.8 \times 10^{-13}$\\
NF1 & Pulmonic stenosis [HP:0001642] & p.Arg1830 & 13/105 (12\%) & $^{\complement}$ & 5/162 (3\%) & 0.005 & 0.013\\
NF1 & Scoliosis [HP:0002650] & p.Arg1830 & 8/125 (6\%) & $^{\complement}$ & 48/240 (20\%) & $4.0 \times 10^{-04}$ & $1.0 \times 10^{-03}$\\
PTPN11 & Hypertelorism [HP:0000316] & missense & 37/41 (90\%) & $^{\complement}$ & 0/12 (0\%) & $6.8 \times 10^{-09}$ & $2.7 \times 10^{-08}$\\
PTPN11 & Intellectual disability, mild [HP:0001256] & missense & 8/23 (35\%) & $^{\complement}$ & 0/12 (0\%) & 0.032 & 0.032\\
PTPN11 & Pulmonic stenosis [HP:0001642] & missense & 18/34 (53\%) & $^{\complement}$ & 0/12 (0\%) & $1.0 \times 10^{-03}$ & 0.002\\
PTPN11 & Webbed neck [HP:0000465] & missense & 15/20 (75\%) & $^{\complement}$ & 0/12 (0\%) & $2.9 \times 10^{-05}$ & $5.9 \times 10^{-05}$\\
RPGRIP1 & Eye poking [HP:0001483] & 1107del/1107del OR 1107del/$^{\complement}$ & 16/16 (100\%) & $^{\complement}$/$^{\complement}$ & 19/41 (46\%) & $1.3 \times 10^{-04}$ & 0.002\\
SAMD9L & Neutropenia [HP:0001875] & Arg986Cys & 7/9 (78\%) & Ser626Leu & 0/9 (0\%) & 0.002 & 0.008\\
SAMD9L & Pancytopenia [HP:0001876] & Arg986Cys & 4/6 (67\%) & Ser626Leu & 0/9 (0\%) & 0.011 & 0.026\\
SAMD9L & Thrombocytopenia [HP:0001873] & Arg986Cys & 7/9 (78\%) & Ser626Leu & 0/9 (0\%) & 0.002 & 0.008\\
SATB2 & Cleft palate [HP:0000175] & missense & 11/49 (22\%) & $^{\complement}$ & 59/105 (56\%) & $1.1 \times 10^{-04}$ & 0.002\\
SCN2A & Intellectual disability [HP:0001249] & I repeat & 21/42 (50\%) & $^{\complement}$ & 157/190 (83\%) & $2.6 \times 10^{-05}$ & $4.2 \times 10^{-04}$\\
SCN2A & Neurodevelopmental abnormality [HP:0012759] & I repeat & 48/65 (74\%) & $^{\complement}$ & 198/218 (91\%) & $1.0 \times 10^{-03}$ & 0.009\\
SCN2A & Autism [HP:0000717] & missense & 59/146 (40\%) & $^{\complement}$ & 33/43 (77\%) & $2.7 \times 10^{-05}$ & $8.6 \times 10^{-05}$\\
SCN2A & Focal-onset seizure [HP:0007359] & missense & 141/170 (83\%) & $^{\complement}$ & 8/33 (24\%) & $8.7 \times 10^{-11}$ & $6.9 \times 10^{-10}$\\
SCN2A & Generalized-onset seizure [HP:0002197] & missense & 104/133 (78\%) & $^{\complement}$ & 6/31 (19\%) & $1.4 \times 10^{-09}$ & $5.7 \times 10^{-09}$\\
SCN2A & Intellectual disability [HP:0001249] & missense & 144/198 (73\%) & $^{\complement}$ & 34/34 (100\%) & $9.4 \times 10^{-05}$ & $2.5 \times 10^{-04}$\\
SCN2A & Motor seizure [HP:0020219] & missense & 146/175 (83\%) & $^{\complement}$ & 6/31 (19\%) & $4.5 \times 10^{-12}$ & $7.2 \times 10^{-11}$\\
SCN2A & Neurodevelopmental abnormality [HP:0012759] & missense & 201/238 (84\%) & $^{\complement}$ & 45/45 (100\%) & $1.0 \times 10^{-03}$ & 0.003\\
SCN2A & Seizure [HP:0001250] & missense & 298/327 (91\%) & $^{\complement}$ & 28/53 (53\%) & $1.6 \times 10^{-10}$ & $8.4 \times 10^{-10}$\\
SCO2 & Hypertrophic cardiomyopathy [HP:0001639] & p.Glu140Lys/ p.Glu140Lys & 2/6 (33\%) & $^{\complement}$/$^{\complement}$ OR p.Glu140Lys/$^{\complement}$ & 13/13 (100\%) & 0.004 & 0.027\\
SETD2 & Delayed ability to walk [HP:0031936] & p.Arg1740Trp & 8/8 (100\%) & $^{\complement}$ & 1/10 (10\%) & $4.1 \times 10^{-04}$ & 0.003\\
SETD2 & Hypertelorism [HP:0000316] & p.Arg1740Trp & 11/11 (100\%) & $^{\complement}$ & 5/23 (22\%) & $1.5 \times 10^{-05}$ & $2.6 \times 10^{-04}$\\
SETD2 & Macrocephaly [HP:0000256] & p.Arg1740Trp & 0/11 (0\%) & $^{\complement}$ & 19/28 (68\%) & $1.5 \times 10^{-04}$ & 0.002\\
SETD2 & Scoliosis [HP:0002650] & p.Arg1740Trp & 6/6 (100\%) & $^{\complement}$ & 2/14 (14\%) & $7.2 \times 10^{-04}$ & 0.005\\
SETD2 & Severe global developmental delay [HP:0011344] & p.Arg1740Trp & 9/9 (100\%) & $^{\complement}$ & 0/12 (0\%) & $3.4 \times 10^{-06}$ & $1.2 \times 10^{-04}$\\
SETD2 & Ventriculomegaly [HP:0002119] & p.Arg1740Trp & 4/4 (100\%) & $^{\complement}$ & 2/17 (12\%) & 0.003 & 0.012\\
SETD2 & Wide nasal bridge [HP:0000431] & p.Arg1740Trp & 9/9 (100\%) & $^{\complement}$ & 2/9 (22\%) & 0.002 & 0.012\\
SETD2 & Macrocephaly [HP:0000256] & missense & 4/24 (17\%) & $^{\complement}$ & 15/15 (100\%) & $1.5 \times 10^{-07}$ & $6.3 \times 10^{-06}$\\
SMAD3 & Osteoarthritis [HP:0002758] & p.Arg287Trp & 19/19 (100\%) & $^{\complement}$ & 7/19 (37\%) & $3.7 \times 10^{-05}$ & $8.6 \times 10^{-04}$\\
SMARCB1 & Atypical teratoid/rhabdoid tumor [HP:0034401] & SV & 8/9 (89\%) & $^{\complement}$ & 2/19 (11\%) & $1.2 \times 10^{-04}$ & $5.9 \times 10^{-04}$\\
SMARCB1 & Embryonal neoplasm [HP:0002898] & SV & 8/8 (100\%) & $^{\complement}$ & 2/19 (11\%) & $2.0 \times 10^{-05}$ & $1.5 \times 10^{-04}$\\
SMARCB1 & Neoplasm by anatomical site [HP:0011793] & SV & 3/3 (100\%) & $^{\complement}$ & 3/20 (15\%) & 0.011 & 0.028\\
SMARCB1 & Neoplasm by histology [HP:0011792] & SV & 11/11 (100\%) & $^{\complement}$ & 4/21 (19\%) & $1.1 \times 10^{-05}$ & $1.5 \times 10^{-04}$\\
SMARCB1 & Neuroepithelial neoplasm [HP:0030063] & SV & 2/2 (100\%) & $^{\complement}$ & 0/17 (0\%) & 0.006 & 0.018\\
SMARCB1 & Rhabdoid tumor [HP:0034557] & SV & 4/4 (100\%) & $^{\complement}$ & 2/19 (11\%) & 0.002 & 0.006\\
SMARCC2 & Intellectual disability [HP:0001249] & c.3222del & 1/6 (17\%) & $^{\complement}$ & 49/52 (94\%) & $7.0 \times 10^{-05}$ & 0.006\\
SPTAN1 & Appendicular spasticity [HP:0034353] & Arg19Trp & 21/21 (100\%) & $^{\complement}$ & 2/15 (13\%) & $4.5 \times 10^{-08}$ & $1.9 \times 10^{-07}$\\
SPTAN1 & Distal lower limb muscle weakness [HP:0009053] & Arg19Trp & 8/15 (53\%) & $^{\complement}$ & 1/20 (5\%) & 0.002 & 0.004\\
SPTAN1 & Epileptic spasm [HP:0011097] & Arg19Trp & 0/19 (0\%) & $^{\complement}$ & 15/20 (75\%) & $7.7 \times 10^{-07}$ & $2.7 \times 10^{-06}$\\
SPTAN1 & Infantile spasms [HP:0012469] & Arg19Trp & 0/19 (0\%) & $^{\complement}$ & 15/41 (37\%) & $1.0 \times 10^{-03}$ & 0.003\\
SPTAN1 & Intellectual disability [HP:0001249] & Arg19Trp & 0/21 (0\%) & $^{\complement}$ & 27/34 (79\%) & $1.8 \times 10^{-09}$ & $1.5 \times 10^{-08}$\\
SPTAN1 & Lower limb muscle weakness [HP:0007340] & Arg19Trp & 14/14 (100\%) & $^{\complement}$ & 17/30 (57\%) & 0.003 & 0.007\\
SPTAN1 & Lower limb spasticity [HP:0002061] & Arg19Trp & 21/21 (100\%) & $^{\complement}$ & 0/13 (0\%) & $1.1 \times 10^{-09}$ & $1.4 \times 10^{-08}$\\
SPTAN1 & Microcephaly [HP:0000252] & Arg19Trp & 0/21 (0\%) & $^{\complement}$ & 19/45 (42\%) & $2.4 \times 10^{-04}$ & $6.1 \times 10^{-04}$\\
SPTAN1 & Motor axonal neuropathy [HP:0007002] & Arg19Trp & 0/16 (0\%) & $^{\complement}$ & 12/19 (63\%) & $6.3 \times 10^{-05}$ & $1.7 \times 10^{-04}$\\
SPTAN1 & Motor seizure [HP:0020219] & Arg19Trp & 0/19 (0\%) & $^{\complement}$ & 20/25 (80\%) & $3.3 \times 10^{-08}$ & $1.7 \times 10^{-07}$\\
SPTAN1 & Peripheral axonal neuropathy [HP:0003477] & Arg19Trp & 4/20 (20\%) & $^{\complement}$ & 12/19 (63\%) & 0.010 & 0.017\\
SPTAN1 & Seizure [HP:0001250] & Arg19Trp & 2/21 (10\%) & $^{\complement}$ & 35/40 (88\%) & $2.4 \times 10^{-09}$ & $1.5 \times 10^{-08}$\\
SPTAN1 & Spastic paraplegia [HP:0001258] & Arg19Trp & 21/21 (100\%) & $^{\complement}$ & 0/20 (0\%) & $3.7 \times 10^{-12}$ & $9.3 \times 10^{-11}$\\
SPTAN1 & Spasticity [HP:0001257] & Arg19Trp & 21/21 (100\%) & $^{\complement}$ & 5/18 (28\%) & $1.0 \times 10^{-06}$ & $3.3 \times 10^{-06}$\\
SPTAN1 & Appendicular spasticity [HP:0034353] & missense & 21/22 (95\%) & $^{\complement}$ & 2/14 (14\%) & $8.7 \times 10^{-07}$ & $8.1 \times 10^{-06}$\\
SPTAN1 & Distal lower limb muscle weakness [HP:0009053] & missense & 9/19 (47\%) & $^{\complement}$ & 0/16 (0\%) & $1.0 \times 10^{-03}$ & 0.004\\
SPTAN1 & Epileptic spasm [HP:0011097] & missense & 2/22 (9\%) & $^{\complement}$ & 13/17 (76\%) & $2.9 \times 10^{-05}$ & $1.4 \times 10^{-04}$\\
SPTAN1 & Infantile spasms [HP:0012469] & missense & 2/29 (7\%) & $^{\complement}$ & 13/31 (42\%) & 0.002 & 0.005\\
SPTAN1 & Intellectual disability [HP:0001249] & missense & 9/32 (28\%) & $^{\complement}$ & 18/23 (78\%) & $3.4 \times 10^{-04}$ & $1.0 \times 10^{-03}$\\
SPTAN1 & Lower limb muscle weakness [HP:0007340] & missense & 15/16 (94\%) & $^{\complement}$ & 16/28 (57\%) & 0.015 & 0.030\\
SPTAN1 & Lower limb spasticity [HP:0002061] & missense & 21/22 (95\%) & $^{\complement}$ & 0/12 (0\%) & $2.4 \times 10^{-08}$ & $6.6 \times 10^{-07}$\\
SPTAN1 & Microcephaly [HP:0000252] & missense & 4/34 (12\%) & $^{\complement}$ & 15/32 (47\%) & 0.002 & 0.005\\
SPTAN1 & Motor axonal neuropathy [HP:0007002] & missense & 0/19 (0\%) & $^{\complement}$ & 12/16 (75\%) & $2.2 \times 10^{-06}$ & $1.5 \times 10^{-05}$\\
SPTAN1 & Motor seizure [HP:0020219] & missense & 7/27 (26\%) & $^{\complement}$ & 13/17 (76\%) & 0.002 & 0.004\\
SPTAN1 & Peripheral axonal neuropathy [HP:0003477] & missense & 4/23 (17\%) & $^{\complement}$ & 12/16 (75\%) & $6.7 \times 10^{-04}$ & 0.002\\
SPTAN1 & Seizure [HP:0001250] & missense & 13/33 (39\%) & $^{\complement}$ & 24/28 (86\%) & $2.5 \times 10^{-04}$ & $9.9 \times 10^{-04}$\\
SPTAN1 & Spastic paraplegia [HP:0001258] & missense & 21/25 (84\%) & $^{\complement}$ & 0/16 (0\%) & $4.7 \times 10^{-08}$ & $6.6 \times 10^{-07}$\\
SPTAN1 & Spasticity [HP:0001257] & missense & 22/23 (96\%) & $^{\complement}$ & 4/16 (25\%) & $5.2 \times 10^{-06}$ & $2.9 \times 10^{-05}$\\
SPTAN1 & Appendicular spasticity [HP:0034353] & truncating & 0/8 (0\%) & $^{\complement}$ & 23/28 (82\%) & $4.3 \times 10^{-05}$ & $2.7 \times 10^{-04}$\\
SPTAN1 & Hypotonia [HP:0001252] & truncating & 1/11 (9\%) & $^{\complement}$ & 15/24 (62\%) & 0.004 & 0.015\\
SPTAN1 & Lower limb spasticity [HP:0002061] & truncating & 0/8 (0\%) & $^{\complement}$ & 21/26 (81\%) & $7.1 \times 10^{-05}$ & $3.5 \times 10^{-04}$\\
SPTAN1 & Motor axonal neuropathy [HP:0007002] & truncating & 12/12 (100\%) & $^{\complement}$ & 0/23 (0\%) & $1.2 \times 10^{-09}$ & $3.0 \times 10^{-08}$\\
SPTAN1 & Peripheral axonal neuropathy [HP:0003477] & truncating & 12/12 (100\%) & $^{\complement}$ & 4/27 (15\%) & $4.6 \times 10^{-07}$ & $5.8 \times 10^{-06}$\\
SPTAN1 & Spastic paraplegia [HP:0001258] & truncating & 0/8 (0\%) & $^{\complement}$ & 21/33 (64\%) & $1.0 \times 10^{-03}$ & 0.005\\
SPTAN1 & Spasticity [HP:0001257] & truncating & 0/8 (0\%) & $^{\complement}$ & 26/31 (84\%) & $2.1 \times 10^{-05}$ & $1.7 \times 10^{-04}$\\
SUOX & Microcephaly [HP:0000252] & homodimerization/ homodimerization OR homodimerization/$^{\complement}$ & 0/9 (0\%) & $^{\complement}$/$^{\complement}$ & 10/12 (83\%) & $2.2 \times 10^{-04}$ & 0.003\\
TBCK & Developmental regression [HP:0002376] & R126*/R126* & 9/12 (75\%) & R126*/$^{\complement}$ OR $^{\complement}$/$^{\complement}$ & 2/22 (9\%) & $1.8 \times 10^{-04}$ & 0.003\\
TBCK & Macroglossia [HP:0000158] & R126*/R126* & 11/12 (92\%) & R126*/$^{\complement}$ OR $^{\complement}$/$^{\complement}$ & 3/22 (14\%) & $1.3 \times 10^{-05}$ & $4.3 \times 10^{-04}$\\
TBX1 & Global developmental delay [HP:0001263] & Tyr418PhefsTer42 & 5/5 (100\%) & $^{\complement}$ & 3/20 (15\%) & $1.0 \times 10^{-03}$ & 0.023\\
TBX1 & Narrow nose [HP:0000460] & Tyr418PhefsTer42 & 5/5 (100\%) & $^{\complement}$ & 0/6 (0\%) & 0.002 & 0.024\\
TBX5 & Upper limb phocomelia [HP:0009813] & Arg237Gln & 7/22 (32\%) & $^{\complement}$ & 3/131 (2\%) & $4.6 \times 10^{-05}$ & $7.3 \times 10^{-04}$\\
TBX5 & Ventricular septal defect [HP:0001629] & Arg237Gln & 0/17 (0\%) & $^{\complement}$ & 61/73 (84\%) & $5.6 \times 10^{-11}$ & $1.8 \times 10^{-09}$\\
TBX5 & Ventricular septal defect [HP:0001629] & missense & 31/60 (52\%) & $^{\complement}$ & 30/30 (100\%) & $4.6 \times 10^{-07}$ & $1.5 \times 10^{-05}$\\
TGFBR1 & Self-healing squamous epithelioma [HP:0034720] & Gly52Arg & 7/7 (100\%) & $^{\complement}$ & 11/33 (33\%) & 0.002 & 0.012\\
TGFBR1 & Arterial tortuosity [HP:0005116] & MSSE var & 1/19 (5\%) & $^{\complement}$ & 8/16 (50\%) & 0.005 & 0.026\\
TGFBR1 & Hypertelorism [HP:0000316] & MSSE var & 0/18 (0\%) & $^{\complement}$ & 15/19 (79\%) & $5.0 \times 10^{-07}$ & $4.0 \times 10^{-06}$\\
TGFBR1 & Self-healing squamous epithelioma [HP:0034720] & MSSE var & 18/19 (95\%) & $^{\complement}$ & 0/21 (0\%) & $1.7 \times 10^{-10}$ & $2.7 \times 10^{-09}$\\
UMOD & Hyperuricemia [HP:0002149] & EGF & 14/32 (44\%) & $^{\complement}$ & 50/57 (88\%) & $1.8 \times 10^{-05}$ & $1.1 \times 10^{-04}$\\
UMOD & Hyperuricemia [HP:0002149] & cysteine & 38/41 (93\%) & $^{\complement}$ & 26/48 (54\%) & $4.5 \times 10^{-05}$ & $2.7 \times 10^{-04}$\\
WWOX & Bilateral tonic-clonic seizure with focal onset [HP:0007334] & MAPT Interaction/MAPT Interaction OR MAPT Interaction/$^{\complement}$ & 0/18 (0\%) & $^{\complement}$/$^{\complement}$ & 7/9 (78\%) & $4.1 \times 10^{-05}$ & 0.002\\
ZFX & Hyperparathyroidism [HP:0000843] & missense & 7/9 (78\%) & $^{\complement}$ & 0/5 (0\%) & 0.021 & 0.021\\
ZMYM3 & Cupped ear [HP:0000378] & R441 & 7/10 (70\%) & $^{\complement}$ & 1/23 (4\%) & $2.0 \times 10^{-04}$ & 0.012\\
\hline
\end{longtable}
\end{scriptsize}
\end{center}
%%%%%%%%%%%%%%%%%%%%%%%%%%%%%%%%%%%%%%
%%%%%%%%%%%%%%%%%%%%%%%%%%%%%%%%%%%%%%%%%%%%%%%%%%%%%%%%%%%%%%%%%%%
\clearpage
\newpage

\begin{table}
\centering
\toprule
\begin{tabular}{l>{\raggedright}p{2.5cm}>{\raggedright}p{2.5cm}l>{\raggedright}p{2.5cm}l}
\textbf{cohort} & \textbf{genotype (A)} & \textbf{genotype (B)} & \textbf{Scorer} & \textbf{p-val} & \textbf{xrefs}\\
\midrule
ANKRD11 & FEMALE & MALE & HPO Group Count & 0.007 & - \\
ANKRD11	& SV & $^{\complement}$ & HPO Group Count & $2.64\times 10^{-4}$ & \cite{PMID_36446582} \\
CHD8 & FEMALE & MALE & De Vries Score & 0.006 & - \\
CHD8 & missense & $^{\complement}$ & De Vries Score & 	$8.99\times 10^{-04}$ & - \\
CTCF & missense & $^{\complement}$ & De Vries Score & 0.009 & - \\ 
LMNA & Upstream of NLS & $^{\complement}$ & HPO Group Count & $1.83\times 10^{-16}$ & - \\
RERE & LoF & Atrophin & HPO Group Count & 0.001 & -  \\
\hline
\end{tabular}
\caption{Phenotype severity scores. Mann-Whitney U tests performed using GPSEA to assess the association between a genotype and the total value of a phenotype severity score.
The references in the xrefs column show previous publications that have presented similar findings. HPO Group Count scorer assigns a phenotype score that is equivalent to the count of present phenotypes that are either an exact match to the query terms or their descendants. DeVries scorer is an adaption of the DeVries score \cite{PMID_34521999} using HPO. $\complement$: set complement of a variant predicate. See Table \ref{tab:hpo_fet} for a definition.} 
\label{tab:sevscores}
\end{table}


%%%%%%%%%%%%%%%%%%%%%%%%%%%%%%%%%%%%%%
\clearpage
\newpage


\begin{table}
\centering
\begin{tabular}{l>{\raggedright\arraybackslash}p{3cm}>{\raggedright\arraybackslash}p{2.5cm}>{\raggedright\arraybackslash}p{4.5cm}lr}
\toprule
\textbf{cohort} & \textbf{genotype (A)} & \textbf{genotype (B)} & \textbf{Outcome Variable} & \textbf{p-val} & \textbf{xrefs}\\
\midrule
ACADM & K329Q: 1/1 &  $1/\complement$ OR $\complement/\complement$ & MCAD Activity\% [LOINC:74892-1] & $6.1 \times 10^{-10}$ & \cite{PMID_33580884}\\
ACADM & Y67H: 1/1 OR $1/\complement$  & $\complement/\complement$ & MCAD Activity\% [LOINC:74892-1] & $2.0 \times 10^{-05}$ & \cite{PMID_33580884}\\
CYP21A2 & missense:  $1/\complement$  OR $\complement/\complement$ &  1/1 & 17-Hydroxyprogesterone [LOINC:1668-3] & $7.9 \times 10^{-06}$ & -\\
\bottomrule
\end{tabular}
\caption{\textbf{Student t-tests performed using GPSEA}. $\complement$: set complement of a variant predicate. See Table \ref{tab:hpo_fet} for a definition. Citations in the xrefs column show previous publications that have presented similar findings.}
\label{tab:t_test}
\end{table}
%%%%%%%%%%%%%%%%%%%%%%%%%%%%%%%%%%%%%%%%%%%%%%%%%%%%%%%%%%%%%%%%%%%
\clearpage
\newpage


\begin{table}
\centering
\begin{tabular}{l>{\raggedright\arraybackslash}p{3.5cm}>{\raggedright\arraybackslash}p{2.5cm}>{\raggedright\arraybackslash}p{4cm}lr}
\toprule
\textbf{cohort} & \textbf{genotype (A)} & \textbf{genotype (B)} & \textbf{HPO Term} & \textbf{p-val} & \textbf{xrefs}\\
\midrule
AIRE & R257*/R257* OR R257*/$\complement$ & $\complement$/$\complement$ & Survival analysis: Chronic mucocutaneous candidiasis & 0.019 & -\\
CLDN16 & missense/missense OR missense/$\complement$ & $\complement$/$\complement$ & Survival analysis: Stage 5 chronic kidney disease & 0.034 & -\\
SCO2 & Glu140Lys/Glu140Lys & $\complement$/$\complement$ OR Glu140Lys/$\complement$ & Survival analysis: Hypertrophic cardiomyopathy & 0.219 & -\\
UMOD & 278\_289delins & $\complement$ & Survival analysis: Stage 5 chronic kidney disease & 0.835 & -\\
UMOD & Cys248Trp & Gln316Pro & Survival analysis: Stage 5 chronic kidney disease & $4.1 \times 10^{-04}$ & -\\
UMOD & EGF & $\complement$ & Survival analysis: Stage 5 chronic kidney disease & 0.284 & -\\
\bottomrule
\end{tabular}
\caption{\textbf{Age of onset of phenotypic abnormality}. Logrank tests performed using GPSEA to assess association between a genotype and the 
age of onset of a phenotypic feature represented by an HPO term. $\complement$: set complement of a variant predicate. See Table \ref{tab:hpo_fet} for a definition. Citations in the xrefs column show previous publications that have presented similar findings.}
\label{tab:hpo_onset}
\end{table}
%%%%%%%%%%%%%%%%%%%%%%%%%%%%%%%%%%%%%%%%%%%%%%%%%%%%%%%%%%%%%%%%%%%
\clearpage
\newpage



\begin{table}
\centering
\begin{scriptsize}
\begin{tabular}{l>{\raggedright\arraybackslash}p{4cm}p{2cm}>{\raggedright\arraybackslash}p{2.5cm}lr}
\toprule
\textbf{cohort} & \textbf{genotype (A)} & \textbf{genotype (B)} & \textbf{Disease onset} & \textbf{p-val} & \textbf{xrefs}\\
\midrule
ATP6V0C & missense & $\complement$ & Compute time until OMIM:620465 onset & 0.193 & -\\
CLDN16 & missense/missense OR missense/$\complement$ & $\complement$/$\complement$ & Compute time until OMIM:248250 onset & 0.333 & -\\
CNTNAP2 & 1 allele & 2 alleles & Compute time until OMIM:610042 onset & $6.2 \times 10^{-06}$ & -\\
COL3A1 & missense & $\complement$ & Compute time until OMIM:130050 onset & 0.317 & -\\
FBXL4 & missense/$\complement$ OR $\complement$/$\complement$ & missense/missense & Compute time until OMIM:615471 onset & 0.031 & -\\
HMGCS2 & missense/missense OR missense/$\complement$ & $\complement$/$\complement$ & Compute time until OMIM:605911 onset & 0.038 & -\\
MPV17 & missense/missense OR missense/$\complement$ & $\complement$/$\complement$ & Compute time until OMIM:256810 onset & 0.002 & -\\
SCO2 & Glu140Lys/Glu140Lys & $^{\complement}$/$^{\complement}$ OR Glu140Lys/$\complement$ & Compute time until OMIM:604377 onset & 0.415 & -\\
SETD2 & Missense & $^{\complement}$ & Compute time until OMIM:616831 onset & $8.5 \times 10^{-05}$ & -\\
SUOX & Missense/Missense OR Missense/$^{\complement}$ & $^{\complement}$/$^{\complement}$ & Compute time until OMIM:272300 onset & $9.2 \times 10^{-06}$ & -\\
SUOX & homodimerization/homodimerization OR homodimerization/$^{\complement}$ & $^{\complement}$/$^{\complement}$ & Compute time until OMIM:272300 onset & 0.853 & -\\
\bottomrule
\end{tabular}
\end{scriptsize}
\caption{\textbf{Age of onset of disease}. Log rank tests performed using GPSEA to assess association between a genotype and the age of onset of a disease. 1/1, $1/\complement$, $\complement/\complement$: See Table \ref{tab:hpo_fet} for definitions. Citations in the xrefs column show previous publications that have presented similar findings. \texttt{hdim}: homodimerization;
   \texttt{ OMIM:620465}: Epilepsy, early-onset, 3, with or without developmental delay; 
   \texttt{OMIM:248250}: Hypomagnesemia 3, renal;
   \texttt{OMIM:610042}: Pitt-Hopkins like syndrome 1;
   \texttt{OMIM:130050}: Ehlers-Danlos syndrome, vascular type;
   \texttt{OMIM:615471}: Mitochondrial DNA depletion syndrome 13 (encephalomyopathic type);
   \texttt{OMIM:605911}:  HMG-CoA synthase-2 deficiency;
   \texttt{OMIM:256810}: Mitochondrial DNA depletion syndrome 6 (hepatocerebral type);
   \texttt{OMIM:604377}: Mitochondrial complex IV deficiency, nuclear type 2;
   \texttt{OMIM:616831}: Luscan-Lumish syndrome;
  \texttt{OMIM:272300}: Sulfite oxidase deficiency.
}
\label{tab:disease_onset}
\end{table}

%%%%%%%%%%%%%%%%%%%%%%%%%%%%%%%%%%%%%%%%%%%%%%%%%%%%%%%%%%%%%%%%%%%
\clearpage
\newpage


\begin{table}
\centering
\begin{tabular}{lp{3.5cm}p{3.5cm}>{\raggedright\arraybackslash}p{3cm}lr}
\toprule
\textbf{cohort} & \textbf{genotype (A)} & \textbf{genotype (B)} & \textbf{Disease} & \textbf{p-val} & \textbf{xrefs}\\
\midrule
FBXL4 & missense: $1/\complement$ or $\complement/\complement$& missense: 1/1 & OMIM:615471 & 0.080 & -\\
MPV17 & Pro98Leu: 1/1 & Pro98Leu: $1/\complement$ or $\complement/\complement$ & OMIM:256810 & 0.010 & -\\
NBAS & missense: 1/1 & missense: $1/\complement$ or $\complement/\complement$ & OMIM:614800 & 0.125 & -\\
\bottomrule
\end{tabular}
\caption{\textbf{Age of death}. Log rank tests performed using GPSEA to assess association between a genotype and the age of
    death of individuals with a disease. 1/1, $1/\complement$, $\complement/\complement$: See Table \ref{tab:hpo_fet} for definitions.
    Citations in the xrefs column show previous publications that have presented similar findings.
\texttt{OMIM:615471}: Mitochondrial DNA depletion syndrome 13 (encephalomyopathic type);
\texttt{OMIM:256810}: Mitochondrial DNA depletion syndrome 6 (hepatocerebral type);
\texttt{OMIM:614800}: Short stature, optic nerve atrophy, and Pelger-Huet anomaly.
}
\label{tab:mortality}
\end{table}
%%%%%%%%%%%%%%%%%%%%%%%%%%%%%%%%%%%%%%%%%%%%%%%%%%%%%%%%%%%%%%%%%%%
\clearpage
\newpage

\begin{table}
\centering
\begin{scriptsize}
\begin{tabular}{l>{\raggedright}p{2.5cm}llllrr}
\toprule
\textbf{Cohort} & \textbf{HPO} & \textbf{disease A} & \textbf{} & \textbf{disease B} & \textbf{} & \textbf{p-val} & \textbf{adj. p}\\
\midrule
ANKRD11 & SV & $^{\complement}$ & ANKRD11 phenotypical score(*) & $2.6 \times 10^{-04}$ & \cite{PMID_36446582}\\
ANKRD11 & FEMALE & MALE & ANKRD11 phenotypical score(*) & 0.007 & -\\
CHD8 & missense & $\complement$ & DeVries score & $9.0 \times 10^{-04}$ & -\\
CHD8 & FEMALE & MALE & DeVries score & 0.006 & -\\
CTCF & missense & $\complement$ & DeVries score & 0.009 & -\\
LMNA & NLS Upstream & $\complement$ & Cardiac phenotype score(*)& $1.8 \times 10^{-16}$ & -\\
POGZ & Missense & $\complement$ &  Nagy et al. Phenotypic  score(*)  & 0.429 & \cite{PMID_35052493}\\
POGZ & Missense & $\complement$ &  DeVries score & 0.429 & -\\
RERE & LoF & Atrophin & Jordan et al.  phenotype score(*)  & $1.0 \times 10^{-03}$ & \cite{PMID_29330883}\\
\bottomrule
\end{tabular}
\end{scriptsize}
\caption{\textbf{Fischer exact test for association between disease diagnosis and phenotypic features.}
\texttt{OMIM:606693}: Kufor-Rakeb syndrome (KRS);
 \texttt{OMIM:617225}: Spastic paraplegia 78, autosomal recessive (SPG78);
 \texttt{OMIM:147920}: Kabuki syndrome 1 (KABUK1);
 \texttt{OMIM:300867}: Kabuki syndrome 2 (KABUK2);
 \texttt{OMIM:609192}: Loeys-Dietz syndrome 1 (LDS1);
 \texttt{OMIM:613795}: Loeys-Dietz syndrome 3 (LDS3);
 \texttt{OMIM:619656}: Loeys-Dietz syndrome 6 (LDS6);
 \texttt{OMIM:613826}: Leber congenital amaurosis 6 (LCA6);
 \texttt{OMIM:608194}: Cone-rod dystrophy 13 (CORD13);
 \texttt{OMIM:268310}: Robinow syndrome, autosomal recessive 1 (RRS1);
 \texttt{OMIM:616331}: Robinow syndrome, autosomal dominant 2 (DRS2).
}
\label{tab:disease_dx}
\end{table}


%%%%%%%%%%%%%%%%%%%%%%%%%%%%%%%%%%%%%%%%%%%%%%%%%%%%%%%%%%%%%%%%%%%
\clearpage
\newpage


\begin{table}
\centering
\begin{scriptsize}
\begin{tabular}{l>{\raggedright}p{3cm}lp{2cm}lp{2.4cm}rr}
\toprule
\textbf{cohort} & \textbf{HPO} & \textbf{genotype (A)} & \textbf{Counts (A)} & \textbf{genotype (B)} & \textbf{Counts (B)} & \textbf{p-val} & \textbf{adj. p}\\
\midrule
KDM6A & Intellectual disability, severe [HP:0010864] & FEMALE & 7/25 (28\%) & MALE & 14/18 (78\%) & 0.002 & 0.008\\
\bottomrule
\end{tabular}
\end{scriptsize}
\caption{\textbf{Fischer exact test for association between  between phenotypic features and sex (male, female)}. Adj. p: p value adjusted with Benjamini-Hochberg method.}
\label{tab:mf_hpo}
\end{table}
%%%%%%%%%%%%%%%%%%%%%%%%%%%%%%%%%%%%%%
\clearpage
\newpage


\begin{center}
\begin{longtable}
{lllllllll}
\caption{Detailed results for Table 1 of the main manuscript. \textbf{Sig}: number of significant associations. \textbf{Pub}: Number of these identified in the literature.  \textbf{Categorical}: Association of genotypes with phenotypes by Fisher exact test. \textbf{t test}: Test of means of continuous values by student t test. \textbf{HPO}: Logrank test for association of genotypes with age of onset of a phenotypic abnormality represented by an HPO term. \textbf{Disease}: Logrank test for association of genotypes with age of onset of a disease. \textbf{Mortality}: Logrank test for association of genotypes with age of death. \textbf{Score}: Mann Whitney U test for association of genotypes with magnitude of a phenotype severity score. Cohorts with no found significant associations are not reported.}
\label{tab:all_association_counts} \\
\hline
\multicolumn{1}{c}{\textbf{Cohort}} & \multicolumn{1}{c}{\textbf{Sig}} & \multicolumn{1}{c}{\textbf{Pub}} & \multicolumn{1}{c}{\textbf{Categorical}} & \multicolumn{1}{c}{\textbf{t test}} & \multicolumn{1}{c}{\textbf{HPO}} & \multicolumn{1}{c}{\textbf{Disease}} & \multicolumn{1}{c}{\textbf{Mortality}} & \multicolumn{1}{c}{\textbf{Score}}\\ \hline
\endfirsthead
\multicolumn{9}{c}%

{{\bfseries \tablename\ \thetable{} -- continued from previous page}} \\ 

\hline
\multicolumn{1}{c}{\textbf{Cohort}} & \multicolumn{1}{c}{\textbf{Sig}} & \multicolumn{1}{c}{\textbf{Pub}} & \multicolumn{1}{c}{\textbf{Categorical}} & \multicolumn{1}{c}{\textbf{t test}} & \multicolumn{1}{c}{\textbf{HPO}} & \multicolumn{1}{c}{\textbf{Disease}} & \multicolumn{1}{c}{\textbf{Mortality}} & \multicolumn{1}{c}{\textbf{Score}}\\ \hline
\endhead 
\hline \multicolumn{9}{|r|}{{Continued on next page}} \\ \hline 

\endfoot 
\hline \hline 
\endlastfoot 
ACADM & 2 & 2 & - & 2/2 & - & - & - & -\\
AIRE & 1 & 1 & - & - & 1/1 & - & - & -\\
ANKRD11 & 2 & 1 & - & - & - & - & - & 1/2\\
ATP13A2 & 2 & 2 & 2/2 & - & - & - & - & -\\
BRD4 & 1 & 0 & 0/1 & - & - & - & - & -\\
CHD8 & 2 & 1 & - & - & - & - & - & 1/2\\
CLDN16 & 1 & 1 & - & - & 1/1 & - & - & -\\
CNTNAP2 & 1 & 0 & - & - & - & 0/1 & - & -\\
CTCF & 1 & 0 & - & - & - & - & - & 0/1\\
CYP21A2 & 1 & 1 & - & 1/1 & - & - & - & -\\
EHMT1 & 1 & 1 & 1/1 & - & - & - & - & -\\
FBN1 & 14 & 14 & 14/14 & - & - & - & - & -\\
FBXL4 & 2 & 0 & 0/1 & - & - & 0/1 & - & -\\
FGD1 & 1 & 1 & 1/1 & - & - & - & - & -\\
GLI3 & 14 & 14 & 14/14 & - & - & - & - & -\\
HMGCS2 & 1 & 0 & - & - & - & 0/1 & - & -\\
IKZF1 & 2 & 2 & 2/2 & - & - & - & - & -\\
ITPR1 & 21 & 0 & 0/21 & - & - & - & - & -\\
Kabuki & 2 & 2 & 2/2 & - & - & - & - & -\\
KCNH5 & 1 & 1 & 1/1 & - & - & - & - & -\\
KDM6A & 2 & 1 & 1/2 & - & - & - & - & -\\
LDS 1 and 3 & 4 & 4 & 4/4 & - & - & - & - & -\\
LDS 3 and 6 & 1 & 1 & 1/1 & - & - & - & - & -\\
LMNA & 24 & 24 & 23/23 & - & - & - & - & 1/1\\
MPV17 & 3 & 1 & 0/1 & - & - & 0/1 & 1/1 & -\\
NBAS & 1 & 0 & 0/1 & - & - & - & - & -\\
NF1 & 21 & 21 & 21/21 & - & - & - & - & -\\
PTPN11 & 4 & 0 & 0/4 & - & - & - & - & -\\
RERE & 1 & 1 & - & - & - & - & - & 1/1\\
Robinow syndrome & 4 & 4 & 4/4 & - & - & - & - & -\\
RPGRIP1 & 3 & 3 & 3/3 & - & - & - & - & -\\
SAMD9L & 3 & 0 & 0/3 & - & - & - & - & -\\
SCN2A & 9 & 9 & 9/9 & - & - & - & - & -\\
SCO2 & 1 & 0 & 0/1 & - & - & - & - & -\\
SETD2 & 9 & 8 & 8/8 & - & - & 0/1 & - & -\\
SMAD3 & 1 & 0 & 0/1 & - & - & - & - & -\\
SMARCB1 & 6 & 0 & 0/6 & - & - & - & - & -\\
SMARCC2 & 1 & 1 & 1/1 & - & - & - & - & -\\
SPTAN1 & 35 & 35& 35/35 & - & - & - & - & -\\
SUOX & 2 & 2 & 1/1 & - & - & 0/1 & - & -\\
TBCK & 2 & 0 & 0/2 & - & - & - & - & -\\
TBX1 & 2 & 0 & 0/2 & - & - & - & - & -\\
TBX5  & 3 & 3 & 3/3 & - & - & - & - & -\\
TGFBR1 & 4 & 1 & 1/4 & - & - & - & - & -\\
UMOD & 3 & 3 & 2/2 & - & 1/1 & - & - & -\\
WWOX & 1 & 1 & 1/1 & - & - & - & - & -\\
ZFX & 1 & 1 & 1/1 & - & - & - & - & -\\
ZMYM3 & 1 & 0 & 0/1 & - & - & - & - & -\\
\hline
\end{longtable}
\end{center}

\begin{table}
\centering
\begin{tabular}{>{\raggedright\arraybackslash}p{5.5cm}llll}
\toprule
\textbf{HPO} & \textbf{id} & \textbf{Count} & \textbf{Observed (\%)} & \textbf{Expected (\%)}\\
\midrule
Abnormality of the musculoskeletal system & HP:0033127 & 31 & 20.1\% & 17.9\%\\
Abnormality of limbs & HP:0040064 & 14 & 9.1\% & 11.4\%\\
Abnormality of the nervous system & HP:0000707 & 38 & 24.7\% & 10.8\%\\
Abnormality of metabolism/homeostasis & HP:0001939 & 2 & 1.3\% & 9.5\%\\
Abnormality of the genitourinary system & HP:0000119 & 1 & 0.6\% & 6.5\%\\
Abnormality of the cardiovascular system & HP:0001626 & 11 & 7.1\% & 5.7\%\\
Abnormality of head or neck & HP:0000152 & 10 & 6.5\% & 5.7\%\\
Abnormality of the immune system & HP:0002715 & 5 & 3.2\% & 5.2\%\\
Abnormality of the eye & HP:0000478 & 7 & 4.5\% & 4.6\%\\
Abnormality of the integument & HP:0001574 & 6 & 3.9\% & 4.2\%\\
Abnormality of blood and blood-forming tissues & HP:0001871 & 5 & 3.2\% & 3.5\%\\
Abnormality of the digestive system & HP:0025031 & 2 & 1.3\% & 3.3\%\\
Neoplasm & HP:0002664 & 11 & 7.1\% & 2.8\%\\
Abnormality of the respiratory system & HP:0002086 & 2 & 1.3\% & 2.5\%\\
Abnormality of the endocrine system & HP:0000818 & 1 & 0.6\% & 1.8\%\\
Abnormal cellular phenotype & HP:0025354 & 1 & 0.6\% & 1.2\%\\
Abnormality of the ear & HP:0000598 & 2 & 1.3\% & 1.2\%\\
Abnormality of prenatal development or birth & HP:0001197 & 0 & 0.0\% & 0.9\%\\
Constitutional symptom & HP:0025142 & 0 & 0.0\% & 0.5\%\\
Growth abnormality & HP:0001507 & 5 & 3.2\% & 0.4\%\\
Abnormality of the breast & HP:0000769 & 0 & 0.0\% & 0.2\%\\
Abnormality of the voice & HP:0001608 & 0 & 0.0\% & 0.1\%\\
Abnormality of the thoracic cavity & HP:0045027 & 0 & 0.0\% & 0.0\%\\
\bottomrule
\end{tabular}
\caption{\textbf{Significant HPO terms according to organ system}. Distribution of significant Fisher exact test results according to top-level HPO term. We performed an exact multinomial test to test the null hypothesis that null hypothesis that the observed counts in each top-level HPO term (organ system) follow the proportions that one would expect from the total number of descendent terms of the top-level terms. We obtained a p value of $3.10\times 10^{-31}$, which allows us to accept the alternative hypothesis that at least one of the observed proportions differs from its expected value. The largest differences were observed for \texttt{Abnormality of the nervous system} (expected 10.8\%, observed 24.7\%), \texttt{Abnormality of metabolism/homeostasis} (expected 9.5\%, observed 1.3\%, and \texttt{Neoplasm} (expected 2.8\%, observed 7.1\%).}
\label{tab:to_do}
\end{table}

