\begin{fontsize}
\begin{table}
\centering
\begin{tabular}{lp{2cm}p{2cm}>{\raggedright}p{4.5cm}lr}
\toprule
\textbf{cohort} & \textbf{genotype (A)} & \textbf{genotype (B)} & \textbf{HPO Term} & \textbf{p-val} & \textbf{xrefs}\\
\midrule
AIRE & R257*/R257* OR R257*/other & other/other & Survival analysis: Chronic mucocutaneous candidiasis & 0.019 & -\\
CLDN16 & missense/missense OR missense/other & other/other & Survival analysis: Stage 5 chronic kidney disease & 0.034 & -\\
SCO2 & Glu140Lys/Glu140Lys & other/other OR Glu140Lys/other & Survival analysis: Hypertrophic cardiomyopathy & 0.219 & -\\
UMOD & 278_289delins & other & Survival analysis: Stage 5 chronic kidney disease & 0.835 & -\\
UMOD & Cys248Trp & Gln316Pro & Survival analysis: Stage 5 chronic kidney disease & $4.1 \times 10^{-04}$ & -\\
UMOD & EGF & other & Survival analysis: Stage 5 chronic kidney disease & 0.284 & -\\
\bottomrule
\end{tabular}
\caption{Log rank tests performed using GPSEA to assess association between a genotype and the age of
    onset of a phenotypic feature represented by an HPO term. 1/1: biallelic with reference to the indicate variant; 
    0/1: heterozygous with reference to the indicated variant; 0/0: Neither allele has the indicated variant.
    Citations in the xrefs column show previous publications that have presented similar findings.}
\label{tab:to_do}
\end{table}
\end{fontsize}