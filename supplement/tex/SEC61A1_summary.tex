\begin{figure}[htbp]
\centering
\begin{subfigure}[b]{0.95\textwidth}
\centering
\includegraphics[width=\textwidth]{img/SEC61A1_protein_diagram.pdf} 
\captionsetup{justification=raggedright,singlelinecheck=false}
\caption{Distribution of variants in SEC61A1}
\end{subfigure}

\vspace{2em}

\begin{subfigure}[b]{0.95\textwidth}
\centering
\resizebox{\textwidth}{!}{
\begin{tabular}{llllrr}
\toprule
HPO term & p.Val85Asp & Other variant & p-value & adj. p-value\\
\midrule
Recurrent lower respiratory tract infections [HP:0002783] & 6/6 (100\%) & 0/5 (0\%) & 0.002 & 0.039\\
\bottomrule
\end{tabular}
}
\captionsetup{justification=raggedright,singlelinecheck=false}
\caption{Fisher Exact Test performed to compare HPO annotation frequency with respect to p.Val85Asp and Other variant. Total of
        18 tests were performed.}
\end{subfigure}
\vspace{2em}
\begin{subfigure}[b]{0.95\textwidth}
\centering
\resizebox{\textwidth}{!}{
\begin{tabular}{llllrr}
\toprule
Genotype (A) & Genotype (B) & total tests performed & significant results\\
\midrule
FEMALE & MALE & 24 & 0\\
\bottomrule
\end{tabular}
}
\captionsetup{justification=raggedright,singlelinecheck=false}
\caption{Fisher Exact Test performed to compare HPO annotation frequency with respect to genotypes.}
\end{subfigure}

\vspace{2em}

\caption{The cohort comprised 19 individuals (8 females, 11 males). 1 of these individuals were reported to be deceased. 
A total of 76 HPO terms were used to annotate the cohort. Disease diagnoses: Immunodeficiency, common variable, 15 (OMIM:620670) (11 individuals), 
Tubulointerstitial kidney disease, autosomal dominant, 5 (OMIM:617056) (7 individuals), Neutropenia, severe congenital, 11, autosomal dominant (OMIM:620674) 
(1 individuals). The origin of clinical diversity in patients with SEC61A1 mutation is currently unclear. 
With our present patient set, a particular phenotype cannot be predicted on the basis of location or nature of the mutation \cite{PMID_32325141}. 
A total of 19 unique variant alleles were found in \textit{SEC61A1} (transcript: \texttt{NM\_013336.4}, protein id: \texttt{NP\_037468.1}).}
\end{figure}
