\begin{figure}[htbp]
\section*{ Kabuki}
\centering
\begin{subfigure}[b]{0.95\textwidth}
\centering
\resizebox{\textwidth}{!}{
\begin{tabular}{llllrr}
\toprule
HPO term & OMIM:147920 & OMIM:300867 & p-value & adj. p-value\\
\midrule
Feeding difficulties [HP:0011968] & 8/25 (32\%) & 55/63 (87\%) & $7.41\times 10^{-7}$ & $2.07\times 10^{-5}$\\
Motor delay [HP:0001270] & 4/10 (40\%) & 58/61 (95\%) & $1.04\times 10^{-4}$ & 0.001\\
\bottomrule
\end{tabular}
}
\captionsetup{justification=raggedright,singlelinecheck=false}
\caption{Fisher Exact Test performed to compare HPO annotation frequency with respect to OMIM:147920 and OMIM:300867. Total of
        28 tests were performed. }
\end{subfigure}
\vspace{2em}
\caption{ The cohort comprised 146 individuals (81 females, 65 males). 5 of these individuals were reported to be deceased. 
A total of 200 HPO terms were used to annotate the cohort. Disease diagnoses: Kabuki syndrome 2 (OMIM:300867) (81 individuals), Kabuki Syndrome 1 (OMIM:147920) (65 individuals). 
Previous studies have found that the phenotype of KS2 is different from that of KS1, for example, patients with KS1 type have a higher risk of typical facial features, short stature, 
and frequent infections compared to those with KS2 \cite{PMID_30514738,PMID_23320472,PMID_38373926}.
A total of 146 unique variant alleles were found in \textit{n/a} (transcript: \texttt{n/a}, protein id: \texttt{n/a}).}
\end{figure}
