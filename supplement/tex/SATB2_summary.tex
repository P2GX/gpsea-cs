\begin{figure}[htbp]
\section*{\textit{SATB2}}
\centering
\begin{subfigure}[b]{0.95\textwidth}
\centering
\includegraphics[width=\textwidth]{ img/SATB2_protein_diagram.pdf} 
\captionsetup{justification=raggedright,singlelinecheck=false}
\caption{Distribution of variants in \textit{SATB2}}
\end{subfigure}

\vspace{2em}

\begin{subfigure}[b]{0.95\textwidth}
\centering
\resizebox{\textwidth}{!}{
\begin{tabular}{llllrr}
\toprule
HPO term & Missense & Other & p-value & adj. p-value\\
\midrule
Cleft palate [HP:0000175] & 11/49 (22\%) & 59/105 (56\%) & $1.11\times 10^{-4}$ & 0.002\\
\bottomrule
\end{tabular}
}
\captionsetup{justification=raggedright,singlelinecheck=false}
\caption{         Fisher Exact Test performed to compare HPO annotation frequency with respect to Missense and Other. Total of
        20 tests were performed. }
\end{subfigure}
\vspace{2em}
\begin{subfigure}[b]{0.95\textwidth}
\centering
\resizebox{\textwidth}{!}{
\begin{tabular}{llllrr}
\toprule
Genotype (A) & Genotype (B) & total tests performed & significant results\\
\midrule
ULD & Other & 20 & 0\\
FEMALE & MALE & 20 & 0\\
transcript ablation & other & 20 & 0\\
\bottomrule
\end{tabular}
}
\captionsetup{justification=raggedright,singlelinecheck=false}
\caption{             Fisher Exact Test performed to compare HPO annotation frequency with respect to genotypes. }
\end{subfigure}

\vspace{2em}

\caption{ The cohort comprised 158 individuals (62 females, 90 males, 6 with unknown sex). A total of 11 HPO terms were used to annotate the cohort. Disease diagnosis: Glass syndrome (OMIM:612313). 
Individuals with large chromosomal deletions were diagnosed at earlier ages (mean 2.5 years, $p \leq 0.0006$).
Individuals with missense or disruptive pathogenic variants were more commonly reported to have sialorrhea (p = 0.0115),
and those with large deletions were more likely to have a history of growth retardation (p = 0.0033).
The authors reported a higher prevalence of chronic mucocutanous candidiasis with the variant Arg357Ter
than with other variants. We did not identify a significant difference in prevalence \cite{PMID_29436146}.
A total of 92 unique variant alleles were found in \textit{SATB2} (transcript: \texttt{NM\_001172509.2}, protein id: \texttt{NP\_001165980.1}).}
\end{figure}
