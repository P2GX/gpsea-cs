\begin{figure}[htbp]
\centering
\begin{subfigure}[b]{0.95\textwidth}
\centering
\resizebox{\textwidth}{!}{
\begin{tabular}{llllrr}
\toprule
HPO term & insertion & other & p-value & adj. p-value & ref\\\\
\midrule
Severe global developmental delay [HP:0011344] & 35/41 (85\%) & 0/4 (0\%) & 0.001 & 0.037 & \cite{PMID_38991538} \\
Moderate global developmental delay [HP:0011343] & 6/41 (15\%) & 4/4 (100\%) & 0.001 & 0.037 & \cite{PMID_38991538}\\
\bottomrule
\end{tabular}
}
\captionsetup{justification=raggedright,singlelinecheck=false}
\caption{Fisher Exact Test performed to compare HPO annotation frequency with respect to insertion (n.64\_65insT and three other insertion variants) and other (substition variants). Total of
        53 tests were performed. Note that the results for severe and moderate global developmental delay are symmetrical.}
\end{subfigure}

\vspace{2em}

\begin{subfigure}[b]{0.95\textwidth}
\centering
\resizebox{\textwidth}{!}{
\begin{tabular}{llllrr}
\toprule
Genotype (A) & Genotype (B) & total tests performed & significant results\\
\midrule
n.64\_65insT & other & 53 & 0\\
FEMALE & MALE & 52 & 0\\
\bottomrule
\end{tabular}
}
\captionsetup{justification=raggedright,singlelinecheck=false}
\caption{Fisher Exact Test performed to compare HPO annotation frequency with respect to genotypes.}
\end{subfigure}

\vspace{2em}

\caption{ The cohort comprised 61 individuals (28 females, 33 males). A total of 209 HPO terms were used to annotate the cohort. Disease diagnosis: ReNU syndrome (OMIM:620851). No previous statistical analysis of correlations with PTPN11 missense variants identified in the medical literature. A total of 61 unique variant alleles were found in \textit{RNU4-2} (transcript: \texttt{NR\_003137.3}, protein id: \texttt{}).}
\end{figure}
