\begin{figure}[htbp]
\section*{ LDS 1 and 3}
\centering
\begin{subfigure}[b]{0.95\textwidth}
\centering
\resizebox{\textwidth}{!}{
\begin{tabular}{llllrr}
\toprule
HPO term & OMIM:609192 & OMIM:613795 & p-value & adj. p-value\\
\midrule
Scoliosis [HP:0002650] & 18/21 (86\%) & 20/43 (47\%) & 0.003 & 0.027\\
Hypertelorism [HP:0000316] & 15/19 (79\%) & 13/35 (37\%) & 0.004 & 0.027\\
Aortic aneurysm [HP:0004942] & 11/11 (100\%) & 26/48 (54\%) & 0.004 & 0.027\\
Osteoarthritis [HP:0002758] & 0/11 (0\%) & 26/38 (68\%) & $4.64\times 10^{-5}$ & 0.001\\
\bottomrule
\end{tabular}
}
\captionsetup{justification=raggedright,singlelinecheck=false}
\caption{
Fisher Exact Test performed to compare HPO annotation frequency with respect to OMIM:609192 and OMIM:613795. Total of
24 tests were performed.
}
\end{subfigure}
\vspace{2em}
\caption{ The cohort comprised 90 individuals (29 females, 43 males, 18 with unknown sex). 2 of these individuals were reported to be deceased. A total of 89 HPO terms were used to annotate the cohort. Disease diagnoses: Loeys-Dietz syndrome 3 (OMIM:613795) (49 individuals), Loeys-Dietz syndrome 1 (OMIM:609192) (23 individuals), Multiple self-healing squamous epithelioma, susceptibility to (OMIM:132800) (18 individuals). A total of 37 unique variant alleles were found.}
\end{figure}
