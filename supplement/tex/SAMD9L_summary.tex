\begin{figure}[htbp]
        \section*{ SAMD9L}
        \centering
        \begin{subfigure}[b]{0.95\textwidth}
        \centering
        \includegraphics[width=\textwidth]{ img/SAMD9L_protein_diagram.pdf} 
        \captionsetup{justification=raggedright,singlelinecheck=false}
        \caption{Distribution of variants in SAMD9L}
        \end{subfigure}
        
        \vspace{2em}
        
        \begin{subfigure}[b]{0.95\textwidth}
        \centering
        \resizebox{\textwidth}{!}{
        \begin{tabular}{llllrr}
        \toprule
        HPO term & Arg986Cys & Ser626Leu & p-value & adj. p-value\\
        \midrule
        Pancytopenia [HP:0001876] & 4/6 (67\%) & 0/9 (0\%) & 0.011 & 0.022\\
        Thrombocytopenia [HP:0001873] & 7/9 (78\%) & 0/9 (0\%) & 0.002 & 0.007\\
        Neutropenia [HP:0001875] & 7/9 (78\%) & 0/9 (0\%) & 0.002 & 0.007\\
        \bottomrule
        \end{tabular}
        }
        \captionsetup{justification=raggedright,singlelinecheck=false}
        \caption{         Fisher Exact Test performed to compare HPO annotation frequency with respect to Arg986Cys and Ser626Leu. Total of
                6 tests were performed. }
        \end{subfigure}
        \vspace{2em}
        \begin{subfigure}[b]{0.95\textwidth}
        \centering
        \resizebox{\textwidth}{!}{
        \begin{tabular}{llllrr}
        \toprule
        Genotype (A) & Genotype (B) & total tests performed & significant results\\
        \midrule
        N Term & other & 15 & 0\\
        \bottomrule
        \end{tabular}
        }
        \captionsetup{justification=raggedright,singlelinecheck=false}
        \caption{             Fisher Exact Test performed to compare HPO annotation frequency with respect to genotypes. }
        \end{subfigure}
        
        \vspace{2em}
        
        \caption{ The cohort comprised 31 individuals (15 females, 16 males). 5 of these individuals were reported to be deceased. A total of 41 HPO terms were used to annotate the cohort. Disease diagnoses: Ataxia-pancytopenia syndrome (OMIM:159550) (22 individuals), Spinocerebellar ataxia 49 (OMIM:619806) (9 individuals). A recent summary of SAMD9L variants stated there was no evidence of phenotype–genotype correlation \cite{PMID_38594844}.
         A total of 31 unique variant alleles were found in \textit{SAMD9L} (transcript: \texttt{NM\_152703.5}, protein id: \texttt{NP\_689916.2}).}
        \end{figure}
        