\begin{figure}[htbp]
\section*{Robinow syndrome}
\centering
\begin{subfigure}[b]{0.95\textwidth}
\centering
\resizebox{\textwidth}{!}{
\begin{tabular}{llllrr}
\toprule
HPO term & OMIM:268310 & OMIM:616331 & p-value & adj. p-value\\
\midrule
Mesomelia [HP:0003027] & 31/31 (100\%) & 10/15 (67\%) & 0.002 & 0.048\\
Hearing impairment [HP:0000365] & 3/22 (14\%) & 7/7 (100\%) & $7.69\times 10^{-5}$ & 0.003\\
Short stature [HP:0004322] & 29/29 (100\%) & 3/11 (27\%) & $2.15\times 10^{-6}$ & $1.89\times 10^{-4}$\\
Cleft palate [HP:0000175] & 0/17 (0\%) & 5/8 (62\%) & 0.001 & 0.031\\
\bottomrule
\end{tabular}
}
\captionsetup{justification=raggedright,singlelinecheck=false}
\caption{Fisher Exact Test performed to compare HPO annotation frequency with respect to Robinow syndrome, autosomal recessive (OMIM:268310) and Robinow syndrome, autosomal dominant 2 (OMIM:616331). Total of
        88 tests were performed. }
\end{subfigure}
\vspace{2em}
\caption{
The cohort comprised 48 individuals (10 females, 14 males, 24 with unknown sex).
A total of 103 HPO terms were used to annotate the cohort.
Disease diagnoses: Robinow syndrome, autosomal recessive (OMIM:268310) (32 individuals), Robinow syndrome, autosomal dominant 2 (OMIM:616331) (16 individuals).
Robinow syndrome is a skeletal dysplasia characterized by dysmorphic facial features, short-limbed dwarfism, vertebral segmentation, and genital hypoplasia.
A total of 44 unique variant alleles were found.
}
\end{figure}
